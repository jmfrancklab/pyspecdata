\section{This is an example section}
this is an example

\begin{python}
	# these change
	name = 'CS19_R5100micromolDNP'
	path = '/mnt/bruker/Katherine/nmr/'
	expno = r_[1:13]
	setting = expno.copy()
	dbm = r_[-999,
	  -24.9,
	  -21.4,
	  -19.1,
	  -18.2,
	  -16.6,
	  -15.0,
	  -13.7,
	  -12.6,
	  -11.8,
	  -9.9,
	  -8.9]
	# this should be pretty portable -- maybe stick in fornotebook.py
	integral = integrate_emax(path+name+'/',expno,showimage=True,usephase=True,integration_width=60,usebaseline=True,plotcheckbaseline=False,filter_direct = False,center_peak = True)
	normalizer = integral[where(dbm==-999)[0]].copy()
	if len(normalizer)>1:
	   normalizer = normalizer[0]
	integral /= normalizer
	mask = dbm>-24.
	mask = logical_or(mask,dbm<-99)
	filter_for_highpower = False
	if filter_for_highpower:
	   setting = setting[mask]
	   expno = expno[mask]
	   dbm = dbm[mask]
	   integral = integral[mask]
	figure(2)
	axis('tight')
	lplot(name+'_plot'+thisjobname()+'.pdf',grid=False)
	figure(1)
	axis('tight')
	lplot(name+'_image'+thisjobname()+'.pdf',grid=False)
	setting = expno.copy()
	try:
	   eplots(name+'_emax'+thisjobname(),setting,dbm,expno,integral,plot_slope=False,inverse=True,invalid=[11,12])
	except:
	   print "Couldn't plot, setting ",setting.shape,"expno",expno.shape,"integral",integral.shape,"dbm",dbm.shape
	   raise
\end{python}

\begin{tiny}
\begin{python}
# these change
name = 'oxotempo9mM_100416'
path = '/home/franck/data/cnsi_data/'
power_file = 'mat_data/franck_oxotempo9mM_100416.mat'
# leave the rest of the code relatively consistent
dbm = auto_steps(path+power_file,t_minlength = 60,t_start = 5.5*60)
lplot('powerlog'+thisjobname()+'.pdf')
print ''%(),r'\begin{minipage}{3in}','\npower log:\n\n',', '.join(map((lambda x: r'$%0.2f\;dBm$'%x),dbm)),r'\end{minipage}','\n\n'
dbm = dbm[:-2] # don't use T1 power
dbm = r_[-999,dbm-10] # because I have 10 dB of attenuation immediately before power meter and emax_plot and the code below expect 20

dnp_for_rho(path,name,dbm,expno = r_[5:32],t1expnos = r_[4,33,34])
\end{python}
\end{tiny}
